\section{Режимы работы. Защищенный режим. Системные таблицы.}

\subsection{Три режима работы компьютера на базе процессоров Intel(x86). Адресация аппаратных прерываний в защищенном режиме: таблица дескрипторов прерываний (IDT) – формат дескриптора прерывания, типы шлюзов. Пример заполнения IDT из лабораторной работы.}

\newpage

\subsection{Три режима работы вычислительной системы с архитектурой x86: особенности. Реальный режим: линия А20 – адресное заворачивание. Перевод компьютера в защищенный режим. Линия A20 в защищенном режиме: включение и выключение линии А20 (код из лабораторной работы). XMS.}

\newpage

\subsection{Защищенный режим: назначение системных таблиц – глобальной таблицы дескрипторов (GDT), таблицы дескрипторов прерываний (IDT), теневых регистров (структуры, описывающие дескрипторы GDT и IDT и заполнение дескрипторов в лабораторной работе по защищенному режиму).}

\newpage

\subsection{Защищенный режим: перевод компьютера в защищенный режим – последовательность действий; реализация – пример кода из лабораторной работы.}
