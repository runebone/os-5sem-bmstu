\section{Производство-потребление. Читатели-писатели.}

\subsection{Задача "Производство-потребление": алгоритм Эд. Дейкстры, реализация на семафорах UNIX (код из лабораторной работы).}

\newpage

\subsection{Взаимоисключение и синхронизация процессов и потоков. Семафоры: определение, виды. Семафор, как средство синхронизации и передачи сообщений. Семафоры UNIX: примеры решения задач с помощью семафоров: "Производство-потребление"и "Читатели-писатели"в UNIX (пример реализации в лабораторной работе).}

\newpage

\subsection{Обеспечение монопольного доступа к разделяемым данным в задаче "писатели-читатели": реализация на базе Win32 API (пример кодов лабораторной работы "читатели-писатели"для ОС Windows). (Сравнение мьютексов и семафоров).}

\newpage

\subsection{Задача: читатели-писатели – монитор Хоара, решение с использованием семафоров Unix и разделяемой памяти, пример реализации из лабораторной работы.}

\newpage

\subsection{Взаимодействие параллельных процессов: мониторы – определение; монотор Хоара "читатели-писатели реализация в ОС Windows – пример из лабораторной работы.}

\newpage

\subsection{Взаимодействие параллельных процессов: монопольное использование – реализация; типы реализации взаимодействия. Мониторы – определение, примеры: простой монитор, монитор "кольцевой буфер"и монитор "читатели-писатели". Пример реализации монитора "читатели-писатели"для ОС Windows. Алгорит Э. Дейкстры "Алгоритм банкира"и алгоритм Хабермана с примеров определения состояния системы.}
