\section{Процессы. Общее.}

\subsection{Определение ОС. Ресурсы вычислительной системы. Режимы ядра и задачи: переключение в режим ядра – классификация событий. Процесс, как единица декомпозиции системы, диаграмма состояний процесса с демонстрацией действий, выполняемых в режиме ядра. Переключение контекса. Потоки: типы потоков, особенности каждого типа потоков.}

\newpage

\subsection{Понятие процесса. Процесс как единица декомпозиции системы. Диаграмма состояний процесса с демонстрацией действий, выполняемых в режиме ядра. Планирование и диспетчеризация. Классификация алгоритмов планирования. Примеры алгоритмов планирования, соотнесенные с типами ОС. Процессы и потоки. Типы потоков.}

\newpage

\subsection{ОС с монолитным ядром. Переключение в режим ядра. Диаграмма состояний процесса и переход из одного состояния в другое – причины каждого перехода. Диаграмма состояний процесса в UNIX. Переключение контекста. Система прерываний.}
