\phantomsection\section*{ОПРЕДЕЛЕНИЯ}\addcontentsline{toc}{section}{ОПРЕДЕЛЕНИЯ}

% NOTE lec 1
Компьютер --- это проргаммно-управляемое устройство.
Часть времени работой компьютера управляет ОС, другую часть --- программа.

% Принципы фон Неймана:
% \begin{enumerate}
%     \item Принцип использования двоичной системы счисления. Для представления данных и команд используется двоичная система счисления.
%     \item Принцип программного управления. Работа ЭВМ контролируется программой, состоящей из набора команд. Команды выполняются последовательно друг за другом.
%     \item Принцип однородности памяти. Как команды, так и данные хранятся в одной и той же памяти. Над командами можно выполнять такие же действия, как и над данными.
%     \item Принцип адресуемости памяти. Ячейки памяти ЭВМ имеют адреса, которые последовательно пронумерованы. В любой момент процессору доступна любая ячейка.
%     \item Принцип условного перехода. Не смотря на то, что команды выполняются последовательно, в программах можно реализовать возможность перехода к любому участку кода.
% \end{enumerate}

Принцип хранимой программы --- процессор может выполнять только программу, находящуюся в оперативной памяти.

Мультипрограммность --- в памяти хранится сразу несколько программ и процессор может быстро переключаться с одной на другую.

Мультизадачность --- поддержка загрузки в память нескольких программ.

Терминал --- совокупность клавиатуры и монитора; внешнее устройство; рабочее место программиста.

Квант --- интервал процессорного времени, которое выделено конкретной задаче; главная задача квантования процессорного времени --- обеспечить гарантированное время ответа.

Файловая система --- набор правил, определяющих способ организации, хранения и именования данных на носителях информации.

Вторичная память --- энергонезависимая память, предназначенная для долговременного хранения информации (её главная задача).

Файл --- поименованная совокупность данных, может быть бессмысленная.

Каталог UNIX --- файл в файловой системе.
Содержимое каталога обрабатывается самой операционной системой, а не пользовательской программой.
Каталог содержит файлы, которые в свою очередь могут быть каталогами --- UNIX имеет иерархическую файловую систему.

Ядро --- процессор с полноценным набором регистров (но имеется некоторая виртуализация, так как сейчас процессоры являются программно-управляемыми).

Процесс --- программа в стадии выполнения (программа, которая просто лежит на диске --- это файл).

Планирование --- постановка процессов в очередь (очев., что процессорное время получит процесс, находящийся первым в очереди) по каким-то принципам.

Планировщик --- программа, которая отвечает за порядок получения процессорного времени процессами.

Диспечеризация --- выделение процессорного времени.

Шина --- медный провод.
Если зажать в кулаке 16 проводов, это будет 16-разрядная шина.
Это предполагает параллельную передачу данных.
По каждой шине передаётся один разряд.
В современных компьютерах, работая на определённых частотах, параллельная передача невозможна, только последовательная.

Регистры --- неотъемлемая часть процессора (не память, процессор собственной памяти не имеет).

Порт --- адрес, по которому процессор обращается к внешним устройствам.

Аппаратные прерывания --- абсолютно асинхронные события в системе, так как не зависят от любой работы, выполняемой в системе.
